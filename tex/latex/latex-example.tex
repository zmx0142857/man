% [重要提示]
% 编译本文档的命令是:
% xelatex latex-example.tex
% makeindex latex-example.idx
% xelatex latex-example.tex

% [文档类型]
% ctexart 类用于中文文章, 英文文章则使用 article 类.
% 如果未指定 a4paper, 默认使用的纸张是 usletter, 比 A4 短而宽.
\documentclass[a4paper]{ctexart}

% [导言区]
% 在这里添加需要的包. 多个包可以用逗号隔开放在同一行, 也可以独立成行.
% 建议只将相关的包写在同一行.

% 设置页边距, 如果未使用这个包, 页边距会宽些.
\usepackage{geometry}
% 也可以通过参数自定义页边距
%\usepackage[left=2.5cm,right=2cm,top=2.5cm,bottom=2cm]{geometry}

\usepackage{amsmath}			% 提供 align 等数学环境
\usepackage{amssymb}			% 提供常用符号
\usepackage{mathtools}			% 更多数学环境
\usepackage{bm}					% 粗斜体
\usepackage{mathrsfs}			% 花体

\usepackage{float}				% 浮动体控制 比如 [H]

\usepackage{graphicx}			% 图片
\usepackage{subcaption}			% 组图
\usepackage{wallpaper}          % 背景图片, 封面

\usepackage{tabularx}			% 表格
\usepackage{multirow}			% 多行表格
\usepackage{makecell}			% 单元格内换行
\usepackage{booktabs}			% 在表格内画横线

\usepackage{blindtext}			% 用 \blindtext 添加 balabala 的文字

\usepackage{makeidx}			% 索引
\makeindex

% 自定义颜色
% xcolor 的功能比 color 更强。而且 tikz 包或 pstricks 包要求使用 xcolor
\usepackage{xcolor}
\definecolor{darkblack}{HTML}{2e3436}
\definecolor{lightblack}{HTML}{555753}
\definecolor{darkwhite}{HTML}{d3d7cf}
\definecolor{lightwhite}{HTML}{eeeeec}
\definecolor{darkgreen}{HTML}{4e9a06}
\definecolor{darkpurple}{HTML}{75507b}
\definecolor{darkcyan}{HTML}{06989a}

% 在文档中加入代码
\usepackage{listings}
% 配置代码样式, 注意不能有空行
\lstset{ 
	language=c,						% 语言
	title=\lstname,					% 显示文件名或 caption
	captionpos=b,					% 标题位置: bottom/top/none
	backgroundcolor=\color{lightwhite},	% 背景色
	basicstyle=\ttfamily, 			% 基本样式
	escapeinside={\%*}{*)},			% 在代码中使用 LaTeX
	commentstyle=\color{darkcyan},	% 注释样式
	stringstyle=\color{darkpurple},	% 字符串样式
	% 左 padding, 有行号时勿用
	%xleftmargin=2em,
	%framexleftmargin=2em,
	% 断行
	breakatwhitespace=false,		% 是否在空白符处断行
	breaklines=true,				% 是否自动断行
	% 边框
	frame=single,					% 边框: single/none
	rulecolor=\color{darkwhite},	% 如未设置, 边框颜色可能受彩色文字影响
	% 关键字
	keywordstyle=\color{darkgreen},	% 关键字样式
	morekeywords={*,...},			% 添加关键字
	deletekeywords={...},			% 删除关键字
	% 行号
	numberstyle=\small\color{lightblack},% 行号样式
	numbers=left,					% 行号位置: left/right/none
	%numbersep=10pt,				% 行号与代码的距离
	%firstnumber=1000,				% 开始的行号
	%stepnumber=2,					% 每 n 行显示一个行号
	% 空白符
	tabsize=4,						% tab 大小
	keepspaces=true,				% 保留源代码的缩进 (可能需要 columns=flexible)
	showspaces=false,				% 是否把空格显示成下划线, 覆盖 showstringspaces 属性
	showstringspaces=true,			% 是否在字符串中把空格显示成下划线
	showtabs=false,					% 是否显示字符串中的 tab
}

% 配置各类型链接的颜色
% 引入 hyperref 包, 将生成 pdf 目录, 方便 pdf 阅读器的使用
\usepackage[colorlinks,linkcolor=black,anchorcolor=black,citecolor=black]{hyperref}

% 自定义列表样式
\usepackage{enumitem}
\newlist{subenum}{enumerate}{1}
\setlist[subenum]{label=$(\arabic*)$}

% 自定义定理环境
\newtheorem{theorem}{\hspace{2em}\textbf{定理}}[section]

% 自定义命令, \renewcommand 会覆盖已定义的命令
\renewcommand\d{\mathrm{d}}
\newcommand\dd{\,\mathrm{d}}
\newcommand\e{\mathrm{e}}
\renewcommand\i{\mathrm{i}}

% 如果不喜欢默认字体, 可以自行指定
%\setmainfont{Times New Roman}
%\setCJKmainfont[BoldFont=MicrosoftYaHei, ItalicFont=KaiTi]{SimSun}

% 引入新字体. 通过命令 \Freeserif 使用这个字体
%\newfontfamily\FreeSerif{FreeSerif}

% 令所有 verbatim 环境缩进 2em
\catcode`\@=11
\let \saveverbatime \@xverbatim
\def \@xverbatim {\leftskip = 2em\relax\saveverbatime}
\catcode`\@=12

% 文档信息
\title{\LaTeX 示例 (ctexart)}	% 标题
\author{zmx0142857}				% 作者
%\date{2019-02-20}				% 如不指定则显示编译日期.
								% 不想要显示日期的话, 用 \date{} 即可.

% [文档开始]
\begin{document}

% 标题页
\pagenumbering{gobble}	% 页码格式, 常见的有
						%	gobble - 无页码
						%	arabic - 阿拉伯数字
						%	Roman - 大写罗马数字
						%	roman - 小写罗马数字
						% 此命令应用于页面开头.

\maketitle				% 生成标题页. 如果遗漏, 标题作者等信息将不会出现
\newpage				% 断页

% 目录
\pagenumbering{Roman}
\setcounter{tocdepth}{2}% 目录深度. 1 - 5 分别对应 section, subsection,
						% subsubsection, paragraph, subparagraph.
\tableofcontents		% 生成目录
\newpage

% 正文
\pagenumbering{arabic}

%\section{安装}

\section{入门}
\subsection{编译}

\LaTeX 可以有多种目标文件, 不过最常见的还是生成 \texttt{.pdf} 文件.
编译 \LaTeX 文档的命令是:

\begin{verbatim}
xelatex myfile.tex
\end{verbatim}

编译后一般会产生 \texttt{.aux}, \texttt{.log}, \texttt{.toc} 和
\texttt{.pdf} 文件.  其中 \texttt{.aux} 包含生成引用的信息, \texttt{.log}
记录了编译的详细过程, \texttt{.toc} 则专门用于目录的生成.
如果删除这三种文件, 重新编译时它们仍会出现.

\subsection{基本概念}

\LaTeX \textbf{命令}一般以反斜杠 \verb|\| 开头, 后紧跟一串英文字母.
按参数数目的多少, 常见的 \LaTeX 命令有以下几种形式:

\begin{verbatim}
\command
\command{...}
\command[...]{...}
\command{...}{...}
\end{verbatim}

\verb|\LaTeX|, \verb|\usepackage{geometry}|,
\verb|\documentclass[a4paper]{ctexart}| 都是 \LaTeX 命令.

\LaTeX \textbf{环境}一般以 \verb|\begin{environment}| 开头, 以
\verb|\end{environment}| 结束. 环境可以包含其他的环境, 从而形成树状结构.
事实上整个文档的正文部分都位于 \texttt{document} 环境下.

\subsection{文档结构}

下面是一个极简主义的 \LaTeX 文档示例:

\begin{verbatim}
\documentclass{ctexart}
\begin{document}
    hello world!
\end{document}
\end{verbatim}

尝试拷贝这些代码, 将它们保存成 \texttt{hello-world.tex},
然后编译并查看结果!

\texttt{ctexart} 类用于一篇中文文章, 下面可以有 \texttt{section},
\texttt{subsection}, \texttt{subsubsection} 和 \texttt{paragraph},
\texttt{subparagraph} 等结构.
其中含有 \texttt{sub} 前缀的结构必须直接包含在对应的大结构下.
使用 \texttt{section} 等结构组织文章是很有必要的, 它可以方便地生成目录.

\begin{verbatim}
\documentclass{ctexart}
\begin{document}
一些文字
\section{第一节}
一些文字
\subsection{第一小节}
一些文字
\subsubsection{第一小小节}
一些文字
\paragraph{一个段落}
一些文字
\subparagraph{一个子段落}
一些文字
\section{第二节}
\end{document}
\end{verbatim}

自然段之间用一个空行隔开即可. 当一行文字太长, 而 \LaTeX 不能自动断行时
(比如一个网址), 用 \verb|\\| 手动断行.
手动断行也应用于多行的数学公式或表格中,
大多情况下, 应该尽可能避免手动断行.

\subsection{基本排版}

\paragraph{文字样式} \index{文字样式}

倾斜\textit{italic},
加粗\textbf{boldface},
倾斜并加粗\textbf{\textit{italic-boldface}},
等宽\texttt{typewriter}.

\paragraph{文字对齐}

\begin{flushleft}
	居左 (默认)
\end{flushleft}

\begin{flushright}
	居右
\end{flushright}

\begin{center}
	居中
\end{center}

也可以使用 \verb|\raggedleft, \raggedright, \centering|
来改变整个环境的文字对齐方式 (警告: 不要在 \texttt{document}
环境下直接使用).

\paragraph{注释}

百分号 \texttt{\%} 右边直到行末的内容会被忽略; % 是啊, 会被忽略
\verb|\iffalse| 和 \verb|\fi| 之间的内容会被忽略.

\iffalse
是啊,
会被忽略
\fi

\paragraph{转义符号}

\LaTeX 中的保留字需要转义输出.

\textbackslash \textasciitilde \# \$ \% \^{} \& \{ \} \_ 
$\backslash$

\subsection{列表}

\paragraph{无序列表}

\begin{itemize}
	\item One
	\item Two
\end{itemize}

\paragraph{有序列表}

\begin{enumerate}
	\item One
	\item Two
\end{enumerate}

\paragraph{定制的列表}

使用 \texttt{enumitem} 包, 在导言区完成列表的定制.

\begin{subenum}
	\item One
	\item Two
\end{subenum}

\begin{theorem}
	在定理环境下, 防止列表序号的括号变成斜体的一个方法是将它做成行间公式.
	请看导言区, 注意 \verb|\setlist[subenum]{label=$(\arabic*)$}|
	这行代码中 \verb|$| 的作用.
	\begin{subenum}
	\item One
	\item Two
	\end{subenum}
\end{theorem}

\section{数学环境}

本节有大量内容依赖于 \texttt{amsmath} \index{amsmath} 包和
\texttt{amssymb} 包.  基本上只要写数学文章, 就离不开这两个包.

\subsection{行间公式与独立成行的公式}

行间公式和独立成行的公式各有多种写法, 推荐行间公式写在美元符号之间
\verb|$...$|: $f(x) = \sin^2 x$.  独立成行的公式则推荐用 \verb|\[...\]|:
\[
	f'(x) = 2 \sin x \cos x.
\]
一般情况下独立成行的公式与文字之间没有空行, 表示公式属于这个自然段.

需要公式标号时, 用 \texttt{equation} \index{equation} 环境:
\begin{equation}
	1 + 1 = 2.
\end{equation}

\subsection{\texttt{amsmath} 包提供的数学环境}

\paragraph{多行公式}

如果 \verb|\gather| 带给你麻烦, 也可以连用数个 \verb|\[...\]|
或 \texttt{equation} 环境取而代之.
\begin{gather}
	1 + 1 = 2, \\
	666 = 333 + 333.
\end{gather}

\paragraph{对齐的多行公式}
\begin{align}
	1 + 1 &= 2, \\
	666 &= 333 + 333.
\end{align}

上面几种环境都有各自的不带标号的版本, 如:
\begin{equation*}
	2 + 2 = 4.
\end{equation*}

\subsection{数学环境下的文字样式} \index{文字样式}

\begin{gather*}
	ABCabc123 \quad
	\mathnormal{ABCabc123} \quad
	\mathrm{ABCabc123} \quad
	\mathit{ABCabc123} \quad
	\mathbf{ABCabc123} \quad
	\bm{ABCabc123} \\
	\mathsf{ABCabc123} \quad
	\mathtt{ABCabc123} \quad
	\mathfrak{ABCabc123} \quad
	\mathcal{ABC} \quad
	\mathbb{ABC} \quad
	\mathscr{ABC}
\end{gather*}
\verb|\bm| 需要 \texttt{bm} 包, \verb|\mathscr|
需要 \texttt{mathrsfs} 包.

\subsection{数学环境下的符号}

\paragraph{空白符}

\verb|\!| 可以缩小间距.
\[
	| \! | | \, | \: | \; | \quad | \qquad |
\]

\paragraph{希腊字母}

使用希腊字母的拉丁文名即可.
以大写字母开头的名字表示大写希腊字母.
\[
	\alpha, \pi, \chi,
	\Gamma, \Omega,
	\sigma, \varsigma,
	\epsilon, \varepsilon, \phi, \varphi
\]

\paragraph{比较运算符}
\[
	=, \ne, \approx, >, <, \geq, \geqslant, \leq, \leqslant,
	\equiv, \triangleq, \cong,
	\supset, \subset, \supseteq, \subseteq,
	\sim, \gg, \ll, \unlhd, \lhd
\]

\paragraph{二元运算符}
\[
	+, -, \times, \div, /, \pm, \mp,
	\cup, \cap, \vee, \wedge, \oplus, \otimes
\]

\paragraph{数学函数}
\begin{gather*}
	\sin, \cos, \tan, \cot, \csc, \sec,
	\arcsin, \arccos, \arctan,
	\sinh, \cosh, \tanh, \coth \\
	\gcd, \dim, \ker, \min, \max,
	\log, \ln, \lg
\end{gather*}

\paragraph{积分}
\[
	\int, \iint, \iiint,
	\int \!\!\! \int \!\!\! \int,
	\idotsint, \oint
\]

\paragraph{箭头}
\[
	\rightarrow, \leftarrow, \leftrightarrow,
	\implies, \Leftrightarrow,
	\xrightarrow[ \{1\} {[2]} ]{ [3] \{4\} }
\]

\paragraph{括弧}
\[
	(a), [b], \{c\}, |d|, \|e\|,
	\langle f \rangle,
	\lfloor g \rfloor,
	\lceil h \rceil,
	\ulcorner i \urcorner,
	\left(\frac 1 j \right)
\]

\paragraph{省略号}
\[
	| \dots, \cdots, \ldots, \cdots, \vdots, \ddots |
\]

\begin{enumerate}
	\item dots with commas: $A_1, A_2, \dotsc, A_n$
	\item dots with binary operators: $A_1 + A_2 + \dotsb + A_n$
	\item dots with multiplication: $A_1 A_2 \dotsm A_n$
	\item dots with integrals: $\int \dotsi$
	\item other dots: $A_1 A_2 \dotso A_n$
\end{enumerate}

\paragraph{其他}
\[
	\forall, \exists, \aleph, \in, \notin, \complement,
	\propto, \perp, \parallel, \angle,
	\emptyset, \varnothing, \S, \nabla, \partial,
	\infty, \neg,
	\circ, 90^\circ, \square, \triangle,
\]

注意, 只要你使用 utf-8 编码保存源文件, 并使用 \texttt{xelatex} 编译,
代码中的 unicode 字符一般都能被正常识别 (只要指定了合适的字体).

\subsection{常见的数学表达式}

\paragraph{注音符号}
\[
	x' \quad x'' \quad x''' \quad
	\dot x \quad \ddot x \quad
	\hat x \quad
	\bar x \quad \acute x \quad \check x \quad \grave x \quad
	\breve x \quad \tilde x \quad
	\not x \quad \vec x
\]
\[
	\overrightarrow{AB} \quad \overleftarrow{AB} \quad
	\overline{abc} \quad \underline{abc} \quad
	\widehat{abc} \quad \widetilde{abc} \quad
\]

\paragraph{上下标}

两种先后次序都可以, 但要坚持使用其中一种.
\[
	x_1^3 + x^3_2 \quad
	x_{22}^{10} \quad
	{}_{22}^{10} x
\]

\paragraph{根式}
\[
	\sqrt x \quad \sqrt{x^2+1} \quad \sqrt[n]{n}
\]

\paragraph{分式}

行内分式会缩小: $\frac 1 x$, 但可以强制放大: $\dfrac{1}{x}$. 但是 \LaTeX
的哲学是尽可能使用默认行为, 因此强制放大是不推荐的.
\[
	\frac 1 2 \quad
	\frac{1}{12} \quad
	\frac{1+\sqrt x}{\frac 1 x -x^2} \quad
	1 + \cfrac{1}{2 +
		\cfrac{1}{2 +
		\cfrac{1}{\cdots}}}
\]

\paragraph{大运算符}

其中 \verb|\dd| 是自定义命令.
行间的大运算符会缩小: $\sum_{k=1}^n$, 但可以强制放大:
$\sum\limits_{k=1}^n$. 同上, 不推荐强制放大.
\[
	\sum_{k=1}^n \binom{n}{k} \quad
	\prod_p \frac{1}{p^s} \quad
	\bigcap_{i=1}^n A_i \quad
	\bigcup_{i=1}^n B_i \quad
	\lim_{n \to \infty} \quad
	\sup_{n \geqslant k} \quad
	\inf_{\substack{x < a \\ x \in \mathbb Q}} \quad
	\int_a^b f(x) dx \quad
	\int_a^b f(x) \dd x
\]
\[
	\iiint\limits_\Omega \quad
	\limsup_{n \to \infty} \quad
	\liminf_{n \to \infty} \quad
	\varlimsup_{n \to \infty} \quad
	\varliminf_{n \to \infty}
\]

\paragraph{导数}

其中 \verb|\d| 是自定义命令
\[
	f'(x) \quad f''(x) \quad f'''(x) \quad f^{(n)}(x) \quad
	\frac{du}{dt} \quad
	\frac{\d u}{\d t} \quad
	\frac{\d^2 u}{\d t^2} \quad
	\frac{\partial u}{\partial t}
\]

\paragraph{矩阵}

使用 \texttt{array} 比较麻烦, 不过可以指定每一列的水平对齐方式,
有 \texttt{l, c, r} 三种. 而且 \texttt{array} 可以插入分隔线.
\[
	\begin{matrix}
		-1 &2 \\
		3 &-4 \\
	\end{matrix}
	\quad
	\begin{array}{rr}
		-1 &2 \\
		3 &-4 \\
	\end{array}
	\quad
	\begin{matrix*}[r]
		-1 &2 \\
		3 &-4 \\
	\end{matrix*}
	\quad
	\begin{pmatrix}
		1 &2 \\
		3 &4 \\
	\end{pmatrix}
	\quad
	\begin{bmatrix}
		1 &2 \\
		3 &4 \\
	\end{bmatrix}
	\quad
	\begin{Bmatrix}
		1 &2 \\
		3 &4 \\
	\end{Bmatrix}
	\quad
	\begin{vmatrix}
		1 &2 \\
		3 &4 \\
	\end{vmatrix}
	\quad
	\begin{Vmatrix}
		1 &2 \\
		3 &4 \\
	\end{Vmatrix}
\]
行间矩阵:
$\left[ \begin{smallmatrix} a & b \\ c & d \end{smallmatrix} \right]$.
\[
	\left[
	\begin{array}{c|c}
		1 &2 \\
		\hline
		3 &4 \\
	\end{array}
	\right]
	\quad
	\begin{bmatrix}
		a_{11}	&a_{12}	&\cdots	&a_{1n}	\\
		a_{21}	&a_{22}	&\cdots	&a_{2n}	\\
		\vdots	&\vdots	&\ddots	&\vdots	\\
		a_{m1}	&a_{m2}	&\cdots	&a_{mn} \\
	\end{bmatrix}
	\quad
	\bordermatrix{
		~	&x_1	&x_2	\cr
		A	&1		&0		\cr
		B	&0		&1		\cr
	}
\]

\paragraph{分支公式}
\[
	|x| = \begin{cases}
		x,  &x \geqslant 0; \\
		-x, &x < 0.			\\
	\end{cases}
\]

\paragraph{杂例}
\[
	\frac{
		\begin{matrix*}[r]
			88			\\
			\times 88	\\
		\end{matrix*}
	}{
		7744
	}
	\quad
	\left. \frac{x^3}{3} \right|_0^1
	\quad
	f \big( g(x) \big)
\]

\section{浮动环境与外部文件}

\subsection{图片}

插图只需一条 \verb|\includegraphics| 命令. 文件名可以是绝对路径,
也可以是相对路径. 如果图片和源码在同一目录下, 直接写文件名即可.
\includegraphics[width=0.06\linewidth]{raw/raindrop-adventure.png}

\texttt{figure} 环境是一种浮动环境, 会自行决定插图的位置.
有时会使图片跑到其他段落之间. 使用 \texttt{float} 包的 \verb|H|
选项使图片留在原位置.

\begin{figure}[H]
	\centering
	\includegraphics[width=0.25\linewidth]{raw/raindrop-adventure.png}
	\caption{\texttt{figure} 环境需要 \texttt{graphicx} 包}
\end{figure}

\begin{figure}[H]
	\centering
	\begin{subfigure}[b]{0.25\linewidth}
		\includegraphics[width=\linewidth]{raw/raindrop-adventure.png}
		\caption{左图}
	\end{subfigure}
	\begin{subfigure}[b]{0.25\linewidth}
		\includegraphics[width=\linewidth]{raw/raindrop-adventure.png}
		\caption{右图}
	\end{subfigure}
	\caption{\texttt{subfigure} 环境需要 \texttt{subcaption} 包}
\end{figure}

\subsection{表格}

表格类似于数学环境的 \texttt{array}.
\begin{tabular}{ccc}
	\hline
	a &b &c \\
	\hline
	1 &2 &3 \\
	\hline
\end{tabular}

\begin{table}[H]
\centering
\begin{tabularx}{400pt}{c|X}
	\hline
	1 &balabalabalabalabalabalabalabalabalabalabalabalabalabalabalabalabalabalabala \\
	\hline
	2 &balabalabalabalabalabalabalabalabalabalabalabalabalabalabalabalabalabalabala \\
	\hline
\end{tabularx}
\caption{\texttt{tabularx} 环境需要 \texttt{tabularx} 包}
\label{bala}
\end{table}

跨行单元格 (\texttt{multirow} 包), 单元格内断行 (\texttt{makecell} 包),
适当长度的横线 (\texttt{booktabs} 包):
\begin{center}
\begin{tabular}{ccccc}
	\hline
		\multirow{2}{40pt}{哔哩哔哩哔哩哔哩}
		&\makecell{哔哩\\哔哩}
		&\makecell{哔哩\\哔哩}
		&\makecell{哔哩\\哔哩}
		&\makecell{哔哩\\哔哩} \\
	\cmidrule{2-5}
		&1 &2 &3 &4 \\
	\hline
\end{tabular}
\end{center}

\subsection{代码}

直接写的代码:

\begin{lstlisting}[language=c, caption={hello world}]
/* 
 * output hello world string
 */
#include <stdio.h>
int main()
{
	printf("hello world\n");
	return 0;
}
\end{lstlisting}

外部文件中的代码:
\lstinputlisting[language=c]{raw/hello.c}

注意: 尝试缩小代码的行距 (\verb|basicstyle=\linespread{0.9}\ttfamily|) 将导致背景色出现白色断层.

\subsection{包含外部文件}

\subsubsection{截取或合并 pdf 文件}

引入 \texttt{pdfpages} 包, 在需要插入外部文件的地方添加代码:

\begin{verbatim}
\includepdfmerge{myfile.pdf,10-20}	% 引用该文件的第 10 到 20 页
\end{verbatim}

\subsubsection{引用外部 \LaTeX 文档}

撰写长篇文档时, 常将各章节分别放在不同的 \LaTeX 文档中, 再以
\verb|\include{...}| 命令统一包含进来.
这个命令相当于将外部文件的源码复制到使用命令的位置 (并且自动断页),
因此外部文件不需要 \texttt{document} 环境, 也没有导言区.
另一个命令 \verb|\input{...}| 则不会自动断页.
注意, 指定外部文件时, 不需要写 \texttt{.tex} 后缀, 否则会提示找不到文件.

下面是通过 \verb|\input{...}| 命令与 \texttt{blindtext} 包插入的一段废话.

% 只想看看排版效果时, 就用 blindtext 包来提供一些测试文字
\blindtext
\blindtext


接下来的两页是通过 \verb|\include{...}| 命令加入的.
它们同时也是很好的文档封面的示例.

% 封面页
\begin{center}
	\phantom{空行}
	\vspace{2em} % 用 \vspace 指定垂直距离, 相应的 \hspace 指定水平距离
	\includegraphics[width=0.35\linewidth]{raw/lzu-logo-black.png}
	\vspace{9em}

	% 一般而言, small, large, Large, Huge 等字号已经够用,
	% 不过还是可以自定义文字大小和行高.
	\fontsize{1.2cm}{2.5cm} \textbf{调查报告}
\end{center}

\vspace{5em}

\begin{center}
\Large
\begin{tabular}{cc}
	报告标题			&排版系统: \LaTeX, word 还是 html?				\\
	\cmidrule{2-2}														\\
	作\qquad 者			&某作者											\\
	\cmidrule{2-2}														\\
	单\qquad 位			&某单位											\\
	\cmidrule{2-2}														\\
\end{tabular}
\end{center}
\thispagestyle{empty} % 去掉页眉页脚的文字. 用于页面末尾, \newpage 之前.
\newpage

% 整页的图片, 适合用作衬底/封面
\ThisCenterWallPaper{1}{raw/pinecone.jpg} % 1 表示图片尽可能占满整张纸
\thispagestyle{empty}
\newpage

% 空页. 想让上一页的图片单独显示而不是作为文字的衬底, 就需要空页
\clearpage
\phantom{你看不到我}	% 文字不会显示, 但占据空间
\thispagestyle{empty}
\newpage


\section{引用}

\subsection{标签, 链接与脚注}

\paragraph{标签}

在公式, 定理或浮动环境内使用 \verb|\label{...}|, 再用 \verb|ref{...}|
引用: 表\ref{bala}.

\paragraph{链接}

使用 \texttt{hyperref} 包, 再用 \verb|\href{...}{...}| 引入链接:
\href{https://en.wikibooks.org/wiki/LaTeX}{\LaTeX wikibook},
\href{https://tex.stackexchange.com}{\TeX stackexchange},
\href{https://zmx0142857.github.io/math-note/}{zmx0142857 的数学笔记}.
其中第一个链接是详细的 \LaTeX 的参考文档, 第二个链接是 \LaTeX 问答社区,
第三个是笔者的广告.

\paragraph{脚注}

我右边有个脚注\footnote{我是脚注}.

\subsection{目录与索引}

\paragraph{目录}

一般在文章标题页后面, 正文的前面, 使用 \verb|\tableofcontents| 命令生成.
含有目录的文档, 需要用 \texttt{xelatex} 编译两遍才能得到正确的目录.

\paragraph{索引}

索引一般用于标示术语在文章中出现的位置.
使用索引时, 在导言区添加 \texttt{makeidx} 包并使用 \verb|\makeindex| 命令,
接着在正文中需要索引的位置使用 \verb|\index{...}| 插入索引,
最后在合适的位置 (一般是文章末尾) 用 \verb|\printindex| 列出索引即可.
含有索引的文档应当按下面的步骤编译:

\begin{verbatim}
xelatex myfile.tex
makeindex myfile.idx
xelatex myfile.tex
\end{verbatim}

\subsection{参考文献}

参考文献一般位于文章末尾, 索引的前面.

\paragraph{内嵌参考文献}

如果你只是写一篇短文章, 引用了少量文献, 不想浪费时间在管理参考文献库上,
那么内嵌的方式就足够了. 这里引用了 \cite{book1} 这本书.

\begin{thebibliography}{9}
	\bibitem{book1}
		某作者,
		\emph{书的标题},
		某出版社,
		某年月日.
\end{thebibliography}

\paragraph{使用 \texttt{bibtex}}

这超出了本文档作者的能力范围.

\printindex

% [文档结束]
\end{document}

文档结束后的内容是不可见的.

未添加的内容: 页眉页脚
