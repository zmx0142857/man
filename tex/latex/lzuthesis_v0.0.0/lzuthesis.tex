\documentclass{lzuthesis}

% 中文标题
\zhtitle{兰州大学本科生毕业论文 \LaTeX\ 模板}
% 英文标题
\entitle{Lanzhou University Thesis \LaTeX\ Template for Bachelor Degree}
% 短标题 (用于页眉)
\shorttitle{兰大本科毕设 \LaTeX\ 模板}
% 作者中文名
\zhauthor{王小明}
% 作者英文名
\enauthor{Xiaoming Wang}
% 单位中文名
\zhinstitute{兰州大学数学与统计学院}
% 单位英文名
\eninstitute{School of Mathematics and Statistics, Lanzhou University}
% 导师
\teacher{指导老师}
% 学院
\college{数学与统计学院}
% 专业
\major{专业名称}
% 年级
\grade{2016}

\begin{document}

% 封面
\makecover

% 诚信责任书
\makeliability

% 目录
\pagenumbering{roman}	% 页码为小写罗马数字
\tableofcontents		% 生成目录
\newpage

% 中文摘要
\begin{cnabstract}
	本文主要介绍和讨论了兰州大学数学与统计学院本科毕业论文的 \LaTeX\ 模板.
\end{cnabstract}
\cnkeywords{毕业论文; \LaTeX{}; 模板;}
\newpage

% 英文摘要
\begin{enabstract}
	This thesis is a study on the theory of \dots.
\end{enabstract}
\enkeywords{\LaTeX{};}
\newpage

% 正文
\pagenumbering{arabic} % 页码为阿拉伯数字

\section{关于本文}

本文是根据兰州大学教务处《兰州大学本科毕业论文(设计)指导手册》(2011)
(\url{http://jwc.lzu.edu.cn/upload/doc/N20110111143009.doc},
下称《手册》) 要求所编制的 \LaTeX\ 模板. 具体使用中请根据实际情况更改. 

\section{使用方法}

本模板包含
\begin{itemize}
	\item \texttt{lzuthesis.cls}: 模板文件, 其中定义了本文所用的模板,
		是一个文本文件;
	\item \texttt{lzuthesis.tex}: 论文源文件, 也是文本文件;
	\item \texttt{imgs/}: 文件夹, 包含兰州大学校徽校名的各种版本的矢量 pdf
		文件. 
\end{itemize}

其中封面页, 诚信责任书, 致谢和评语等部分均已在 \texttt{lzuthesis.cls}
中按《手册》要求实现, 而使用方法在 \texttt{lzuthesis.tex} 文件的注释中. 

全文字号字距按照《手册》实现, 如出现不合要求的地方, 请自行使用
\verb|\zihao{n}| 调整字号. 其中 \verb|n = 0| 到 \verb|8| 分别对应初号到八号, \verb|n = -0| 到 \verb|-6| 分别对应小初到小六.

\texttt{lzuthesis.cls} 和 \texttt{lzuthesis.tex}
中的注释与说明应该能够帮助读者进行简单更改.  在编写过程中, 还加入了常见的
\LaTeX\ 宏包, 在编译过程中提示缺少宏包请自行下载. 由于本人能力所限,
\texttt{lzuthesis.cls} 和 \texttt{lzuthesis.tex}
中可能仍然存在着某些缺陷未被发现, 如有更改请加在 \texttt{lzuthesis.cls}
末尾的[修改日志]区域并重新压缩分享. 

推荐编译环境为 TexLive 2019 和 TeXstudio 2.12.6 之后的版本,
在修改了目录相关的结构之后, 编译两次 (TexLive 可通过清华镜像下载,
TeXstudio 为免费软件).

\subsection{定理与证明}

本文提供了常见的定理与证明环境, 如表\ref{table1}

\begin{table}[!h]
\centering
\begin{tabular}{|cc|cc|} \hline
	definition 	& 定义 &
	axiom 		& 公理 \\ \hline
	lemma 		& 引理 &
	theorem 	& 定理 \\ \hline
	proposition & 命题 &
	corollary 	& 推论 \\ \hline
	property 	& 性质 &
	example 	& 例 \\ \hline
	remark 		& 注 & & \\ \hline
	proof		& 证明 &
	solution	& 解 \\ \hline
\end{tabular}
\caption{常见的定理与证明环境}
\label{table1}
\end{table}

\begin{theorem}[\textbf{Banach 不动点定理--压缩映像原理}]
	\label{thm1}
	设 $(\mathscr{X}, \rho)$ 是一个完备的距离空间, $T$ 是 $(\mathscr{X},
	\rho)$ 到其自身的一个压缩映射, 则 $T$ 在 $\mathscr{X}$
	上存在唯一的不动点.
\end{theorem}

\begin{proof}
	略, 详见泛函分析课本. 
\end{proof}

\subsection{图片}

在同一行中, 这里提供了三种图片的排版方式: 一主图, 两主图和两子图.
具体效果如下

\begin{figure}[!h]
	\centering
	\includegraphics[width=0.3\textwidth]{lzu.pdf}
	\caption{正文的图片}
	\label{zheng1}
\end{figure}

\begin{figure}[!h]
	\centering
	\begin{minipage}[t]{0.3\linewidth}
		\centering
		\includegraphics[width=\linewidth]{lzu-blue.pdf}
		\caption{左主图}
	\end{minipage}
	\quad
	\begin{minipage}[t]{0.3\linewidth} 
		\centering 
		\includegraphics[width=\linewidth]{lzu.pdf} 
		\caption{右主图} 
	\end{minipage}
\end{figure}

\begin{figure}[!h]
	\centering
	\begin{subfigure}[b]{0.25\linewidth}
		\includegraphics[width=\linewidth]{lzu-blue.pdf}
		\caption{左子图}
	\end{subfigure}
	\quad
	\begin{subfigure}[b]{0.25\linewidth}
		\includegraphics[width=\linewidth]{lzu.pdf}
		\caption{右子图}
	\end{subfigure}
	\caption{\texttt{subfigure} 环境需要 \texttt{subcaption} 包}
\end{figure}

\subsection{表格}

表格样式可以使用三线表的形式, 具体没做任何要求. 

\begin{table}[!h]
	\centering
	\begin{tabular}{ccc}
		\hline
		a &b &c \\
		\hline
		1 &2 &3 \\
		1 &2 &3 \\
		\hline
	\end{tabular}
	\caption{三线表}
\end{table}

定宽表可以指定表格的宽度. 如果表格中的内容过长, 则会自动换行. 

\begin{table}[!h]
	\centering
	\begin{tabularx}{\linewidth}{c|X}
		\hline
		1 &balabalabalabalabalabalabalabalabalabalabalabalabalabalabalabalabalabalabala
		balabalabalabalabalabalabalabalabalabalabalabalabalabalabalabalabalabalabala \\
		\hline
		2 &balabalabalabalabalabalabalabalabalabalabalabalabalabalabalabalabalabalabala \\
		\hline
	\end{tabularx}
	\caption{\texttt{tabularx} 环境需要 \texttt{tabularx} 包}
	\label{定宽表}
\end{table}

\section{其他}

这里引用附录的图片\ref{fu1}. 

\subsection{推荐}

本模板提供给有一定 \LaTeX\ 基础的同学, 如果需要 \LaTeX\ 入门, 可以参考
\url{https://github.com/zmx0142857/man/} 下的 \verb|tex/| 目录.
此外还有一些常用的 \LaTeX\ 或 \TeX\ 的参考网站
\begin{enumerate}
	\item CTAN, 世界上最主要的 \TeX\ 资源集散网站,
		很多宏包的源码和说明都能查到. (教育网内的镜像站挺多的,
		就不推荐网址了)
	\item \url{https://tex.stackexchange.com/}, \LaTeX\ 和 \TeX\
		的国际交流社区(全英文网站, 需要英语较好)
\end{enumerate}

% 参考文献
\begin{thebibliography}{00}
	\bibitem{r1} 作者. 文章题目 [J].  期刊名, 出版年份,卷号(期数):
		起止页码.
	\bibitem{r2} 作者. 书名 [M]. 版次.
		出版地:出版单位,出版年份:起止页码.
\end{thebibliography}

% 附录
\begin{appendix}
	
\section{附录标题}

《手册》中要求图, 表, 代码等环境在附录中重新开始编号.
如需要更改请修改 \texttt{lzuthesis.cls} 中的 \verb|\setcounter{}| 部分. 

\begin{figure}[!h]
	\centering
	\includegraphics[width=0.3\textwidth]{lzu.pdf}
	\caption{附录的标号从头记数}
	\label{fu1}
\end{figure}

一维坐标系上的热传导方程
\begin{equation}
	\label{eq1}
	\frac{\partial u(x,t)}{\partial t}
	= a\frac{\partial ^{2}u(x,t)}{\partial x^{2}}
\end{equation}

\begin{lstlisting}[language=python, caption={helloworld.py}]
print("Hello World")
\end{lstlisting}
	
\end{appendix}

% 致谢
\begin{acknowledgement}
	感谢爸爸妈妈叔叔阿姨
	感谢爸爸妈妈叔叔阿姨
	感谢爸爸妈妈叔叔阿姨
	感谢爸爸妈妈叔叔阿姨
	感谢爸爸妈妈叔叔阿姨
	感谢爸爸妈妈叔叔阿姨
	感谢爸爸妈妈叔叔阿姨
	感谢爸爸妈妈叔叔阿姨
	感谢爸爸妈妈叔叔阿姨
	感谢爸爸妈妈叔叔阿姨
	感谢爸爸妈妈叔叔阿姨
\end{acknowledgement}

% 评语
\makereviews

\end{document}
