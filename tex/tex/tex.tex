% compile with
% $ tex filename.tex
% a .dvi (DeVice Independent) file will be seen

% disable page numbers
%\nopagenumbers

% magnify the whole document
%\magnification = \magstep 1

\headline = {\TeX\ Example \hfil - \the\pageno\ -}
\footline = {}

\beginsection \TeX\ Example % omit the \ after X and the space will gone

\indent
If one merely wishes to type in ordinary text, without
complicated mathematical formulae or special effects such
as font changes, then one merely has to type it in as it \filbreak
is, leaving a completely blank line between successive
paragraphs.

You do not have to worry about paragraph indentation:
all paragraphs will be indented with the exception of
the first paragraph of a new section.

\beginsection Unit

$$
\vbox{
\settabs
\+ xxx & xxxxxxxxxx & xxxxxxxxxxxxxxxxxxxxxxxxxxxxxxxxxxxxxx \cr
\+ {\tt in} & inch & {\tt 1 in = 25.4 mm = 72 bp = 72.27 pt} \cr
\+ {\tt pt} & point &{\tt 1 pt = 65536 sp} \cr
\+ {\tt pc} & pica &{\tt 1 pc = 12 pt} \cr
\+ {\tt bp} & big point \cr
\+ {\tt cm} & centimetre &{\tt 1 cm = 10 mm} \cr
\+ {\tt mm} & millimetre \cr
\+ {\tt dd} & didot point &{\tt 1157 dd = 1238 pt} \cr
\+ {\tt cc} & cicero &{\tt 1 cc = 12 dd} \cr
\+ {\tt sp} & scaled point \cr
\+ {\tt em} & width of m \cr
\+ {\tt en} & width of n \cr
\+ {\tt ex} & height of x \cr
}$$

\beginsection Layout

$$\vbox{\offinterlineskip
\halign{ \strut \vrule{ \tt #} & \vrule\hfil\ # \hfil & \vrule\ # \vrule \cr
	\noalign{\hrule}
	hsize & text width & 6.5 \cr
	\noalign{\hrule}
	vsize & text height & 8.9 \cr
	\noalign{\hrule}
	hoffset & horizontal offset & 0 (1in-1in on left-up corner) \cr
	\noalign{\hrule}
	voffset & vertical offset & 0 \cr
	\noalign{\hrule}
}
}$$

\beginsection Spaces

\hskip 5em
	This is
		a
 silly
		example of
	a
file with many spaces.

Mr.\ Smith, etc.\ and Proc.\ Amer.\ Math.\ Soc.

Eliminate the%
space with comment!

\vskip 5mm

$| \! | | \, | \> | \; | \quad | \qquad |$.

\beginsection Dashes, Quotes \& Ellipsis

\indent
hyphen-ation; number range 1--100; puctuation dash---like this. $a-b$,
----

`single quote'; ``double quote''; \lq single quote\rq; ``double quote";
''', '\thinspace'', ``\thinspace`, {`}``


$x_1, x_2, \ldots, x_n$, $x_1 + x_2 + \cdots + x_n$, text dots\dots


\beginsection Fonts

\indent
{\rm Roman} {\sl Slanted} {\it Italic} {\tt Typewriter} {\bf Boldface}
\font\sf = cmss10 {\sf Sans Serif}

$
	{\rm Roman}\; {\sl Slanted}\; {\it Italic}\; {\tt Typewriter}\;
	{\bf Boldface}\; {\mit Math Italic}\; {\cal CALLIGRAPHIC}
$

\font\cmrhalf = cmr10 scaled \magstephalf
\font\cmrone = cmr10 scaled \magstep 1
\font\cmrtwo = cmr10 scaled \magstep 2
\font\cmrthree = cmr10 scaled \magstep 3
\font\cmrfour = cmr10 scaled \magstep 4
\font\cmrfive = cmr10 scaled \magstep 5

\centerline{Sample text at magstep 0.}
\centerline{\cmrhalf Sample text at magstephalf.}
\centerline{\cmrone Sample text at magstep 1.}
\centerline{\cmrtwo Sample text at magstep 2.}
\centerline{\cmrthree Sample text at magstep 3.}
\centerline{\cmrfour Sample text at magstep 4.}
\centerline{\cmrfive Sample text at magstep 5.}


\beginsection Accents

\indent
\.a \"a \=a \'a \^a \`a \~a \v a \u a \H a \c c \d a \b a \t aa
\underbar{ABCabc123}

$
	\dot a\; \ddot a\; \bar a\; \acute a\; \hat{a}\; \grave a\;
	\tilde{a}\; \check{a}\; \breve{a}\; \vec{a}\;
	\dot{\dot a}
$

$
	\underline{ABCabc123}\; \overline{ABCabc123}\;
	\underline{\underline{ABCabc123}}
$

\beginsection Miscellaneous Characters

\indent
\i\ \j\ \oe\ \OE\ \ae\ \AE\ \aa\ \AA\ \o\ \O\ \l\ \L\ \ss\ ?` !`
\dag\ \ddag\ \S\ \P\ \copyright\ {\it \$} {\it \&}

$
	\aleph\; \hbar\; \imath\; \jmath\; \ell\; \wp\; \Re\; \Im\;
	\partial\; \infty\; \prime\; \emptyset\; \nabla\; \surd\;
	\top\; \bot\; \|\; \angle\; \triangle\; \forall\; \exists\;
	\neg\; \flat\; \natural\; \sharp\; \clubsuit\; \diamondsuit\;
	\heartsuit\; \spadesuit
$

\beginsection Escape Characters

\indent
\~{} \#\ \$\ \%\ \^{} \&\ \_ \quad
$\#\ \$\ \%\ \&\ \_\ \{\ \}\ \backslash$

\beginsection Lists

\parskip = 0pt
\noindent
Answer all the following questions:
\item{(1)} What is question 1?
\item{(2)} What is question 2?
\item{(3)} What is question 3?
\itemitem{(3a)} What is question 3a?
\itemitem{(3b)} What is question 3b?

\beginsection vrule

Name: \vrule height 0 pt depth 0.4 pt width 2 in

\beginsection Math

\indent
Inline math $f(x) = 3x + 7$, $\root 3 \of {x + 3y}$.
Tilde creates an unbreakable space (nbsp): area~$A$, radius~$r$.

Single line math:
$$
	{numerator \over denominator}, {n \choose k}.
$$

Numbered formulae:
$$
	f(x) = 3x^2 + 19x + 28. \eqno(15)
$$
$$
	M^\bot = \{ f \in V’ : f(m) = 0 \hbox{ for all } m \in M \}.
	\leqno(42)
$$
$$
	\arccos \arcsin \arctan \arg \cos \cosh \cot \coth \csc \deg \det \dim
	\exp \gcd \hom
$$
$$
	\inf \ker \lg \lim \liminf \limsup \ln \log
	\max \min \Pr \sec \sin \sinh \sup \tan \tanh
$$

\beginsection Delimiters

$$
	(), [], \{, \}, |, \|, \left( {a \over b} \right),
	\left. {du \over dx} \right|_{x=0}.
$$

\beginsection Align \& Cases

$$\eqalign{
	\cos 2\theta &= \cos^2 \theta - \sin^2 \theta \cr
				 &= 2 \cos^2 \theta - 1. \cr
}$$
$$\eqalignno{
	\sin 2\theta &= 2\sin \theta \cos \theta, &(6) \cr
	\cos 2\theta &= \cos^2 \theta - \sin^2 \theta \cr
				 &= 2 \cos^2 \theta - 1. &(7) \cr
}$$
$$
	|x| = \cases{
		x &if $x \geq 0$; \cr
		-x &if $x < 0$.\cr
	}
$$

\beginsection Matrices

$$
	\left( \matrix{
		a & b & c \cr
		d & e & f \cr
		g & h & i \cr
	} \right),
	\pmatrix{
		a & b & c \cr
		d & e & f \cr
		g & h & i \cr
	},
	\chi(\lambda) = \left| \matrix{
		\lambda - a & -b & -c \cr
		-d & \lambda - e & -f \cr
		-g & -h & \lambda - i \cr
	} \right|.
$$

\beginsection Theorem

\proclaim Theorem 1 (H.~G.~Wells).
In the country of the blind, the one-eyed man is king.

% start a new page
%\vfill
%\eject

\beginsection Macros

\def\inftyint#1#2{\int_{-\infty}^{+\infty} #1 \,d#2}

$$ \inftyint{f(x)}{x}. $$

\def\frac#1#2{{#1 \over #2}}

$$ \frac{a}{b}. $$

\iffalse
{
	\def\a{\b}
	\def\b{A\def\a{B\def\a{C\def\a{\b}}}}
	\def\puzzle{\a\a\a\a\a}

	\puzzle % ABCAB
}

{
	\def\row #1{(#1_1,\ldots,#1_n)}

	$\row{\bf x}$
	$\row{{\bf x}}$
}

{
	\def\unit #1#2{I must not #1 in #2.}
	\def\two #1{#1 #1}
	\def\ten #1{\two{#1 #1 #1 #1 #1}}
	\def\punishment #1#2{\ten{\ten{\unit{#1}{#2}}}}

	\punishment{run}{the halls}
}

\nullfont balabalabalabala \rm can't see him, ugrrr...

{\tt \string\hi}

{\tt \string\\}

\hbox{T\kern-0.1667em\lower0.5ex\hbox{E}\kern-0.125em X}

\fi

\bye

Exercise to look at:

4.3

7.x
